%%%%%%%%%%%%%%%%%%%%%%%%%%%%%%%%%%%%%%%%%%%%%%%%%%%%%%%%%%%%%%%%%%%%%%%%%%%%%%
\subsection{Equivalent circuits}
%%%%%%%%%%%%%%%%%%%%%%%%%%%%%%%%%%%%%%%%%%%%%%%%%%%%%%%%%%%%%%%%%%%%%%%%%%%%%%

The supercapacitor has a sandwich-like configuration of a collector-electrode-
separator-electrode.  As the current flows from one terminal to the other,
each layer contributes to the overall resistance of the device in the same way
as resistors connected in series.  The goal here is to see if we can assess
the impedance of each individual layer of the sandwich in order to build an
equivalent circuit model that is able to predict the electrical behaviour of the
supercapacitor.  Instead of fitting the parameters of the model to ``reality''
as it is typically done, we would like to infer these from the geometry and 
material properties used in the finite element model.

The electrical resistance of a conductive material exhibiting a uniform cross
section and through which electric current flows uniformly is given by
\begin{equation}
    \text{ESR} = \rho l / A
    \label{eq:pouillets_law}
\end{equation}
Here $l$ measures the length of the material and $A$ is the cross-sectional
area.  The material resitivity $\rho$ is measured in $\ohm\usk\metre$ and is
by definition the inverse of the electrical conductivity.
I use ESR to denote the resistance to avoid confusion with the universal gas
constant $R$.  It stands for equivalent series resistance.

\underline{separator:}
A uniform flow of the electric current through the separator is a reasonable
assumption.  We directly use the formula in Eq.~\eqref{eq:pouillets_law} with 
the computed average resistivity of the separator 
$\langle 1 / \kappa \rangle$.

\underline{collector:}
It is yet unclear whether the assumption is verified in the current
collectors depending on where the tab terminal is located (boundary area over
which we impose the voltage or the current density flowing into the device.
However the conductivity $\sigma$ in the current collector is such that
the contribution of the collectors to the overall resistance of the device is
insignificant (it is several order of magnitude greater than any of the
conductivities in the separator or the electrode).

\underline{electrode:}
The electrode is somewhat thorny.  Whereas in the separator, electric current
is exclusively carried by the ions in the solution, both the matrix and the
solution phases carry a fraction of the current through the electrode.
On top of that, uniformity of the current in each phase over the electrode
domain is out of question since the current is in the solid phase at the
interface with the collector, and, conversely, in the ionic
solution at the other end of the electrode, where the electrode meets with the
separator.
Intuitively though, the electrode resistance must be derivable from the
resistance of the solid and the liquid phases taken separetaly and computed
using $\langle 1 / \sigma \rangle$ and $\langle 1 / \kappa \rangle$,
respectively.

\begin{figure}
    \centering
    \begin{circuitikz}
        \draw (-0.5,0)
        to[short,o-] (0,0)
        to[R,l^=ESR] (2,0)
        to[C,l_=C] (4,0)
        to[short,-o] (5,0);
        \draw (2,0)
        to[short] (2,1)
        to[R,l^=EPR] (4,1)
        to[short] (4,0);
    \end{circuitikz}
    \caption{Equivalent circuit for an electrode.
        ESR = equivalent series resistance.
        EPR = equivalent parallel resistance.
        C   = capacitance.
    }
    \label{fig:electrode_equivalent_circuit}
\end{figure}


The electrode capacitance of an electrode is given by
\begin{equation}
    \text{C} = aC l A
\end{equation}
and its equivalent parallel resistance
\begin{equation}
    \text{EPR} = \frac{RT}{F ai_0 l A}
\end{equation}

\begin{figure}
    \centering
    \begin{circuitikz}
        \draw (-0.5,0)
        to[short,o-] (0,0)
        to[R,l^=R$_1$] (2,0)
        to[generic,l=Y] (2,-2)
        to[R,l_=R$_2$] (4,-2);
        \draw (2,0)
        to[R,l^=R$_1$] (4,0)
        to[generic,l=Y] (4,-2)
        to[R,l_=R$_2$] (6,-2);
        \draw (4,0)
        to[R,l^=R$_1$] (6,0)
        to[generic,l=Y] (6,-2);
        \draw[dashed] (6,0) -- (8,0);
        \draw[dashed] (6,-2) -- (8,-2);
        \draw (8,0)
        to[generic,l=Y] (8,-2)
        to[R,l_=R$_2$] (10,-2);
        \draw (8,0)
        to[R,l^=R$_1$] (10,0)
        to[generic,l=Y] (10,-2)
        to[R,l_=R$_2$] (12,-2)
        to[short,-o] (12.5,-2);
    \end{circuitikz}
    \caption{Ladder.
    With
    R$_1 = \rho_1 dx / A$ ,
    R$_2 = \rho_2 dx / A$ , and
    Y$ = aC dx A$ .
    }
    \label{fig:porous_electrode_as_a_ladder_network}
\end{figure}
This will need some more explanation but in short if we approximate the
electrode by a ladder network as shown in Figure~\ref{fig:porous_electrode_as_a_ladder_network}
and take the low frequency limit, we get
\begin{equation}
    \text{ESR} =
    \frac{1}{3} \frac{(\rho_1 + \rho_2) l}{A}
    + \frac{1}{3} \frac{\rho_1 \rho_2 l}{(\rho_1 + \rho_2) A}
\end{equation}
Here I use $\rho_1$ and $\rho_2$ to designate the electronic and ionic
resistivity of the electrode, respectively $<1/\sigma>$ and $<1/\kappa>$.
