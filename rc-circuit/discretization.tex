%%%%%%%%%%%%%%%%%%%%%%%%%%%%%%%%%%%%%%%%%%%%%%%%%%%%%%%%%%%%%%%%%%%%%%%%%%%%%%
\subsection{Finite element discretization}
%%%%%%%%%%%%%%%%%%%%%%%%%%%%%%%%%%%%%%%%%%%%%%%%%%%%%%%%%%%%%%%%%%%%%%%%%%%%%%

The finite element matrix formulation is derived in the usual fashion by
multiplying the two potentials equations by test functions $\varphi_i$ and
integrating over the whole domain.  It yields
\begin{equation}
    (M + \Delta t K) \Phi^{n+1} = \Delta t f + M \Phi^n
\end{equation}
where the unknown vector $\Phi = (\Phi_1, \Phi_2)^T$ contains the solid phase
and electrolyte potentials at all nodes of the domain triangulation
$\mathcal{T}^h(\Omega)$.  The superscript $n$ denotes the time discretization,
$\Phi^n = \Phi(t^n)$ and $\Delta t = t^{n+1} - t^n$ is the time step.
The approximation of the solution for the solid phase and electrolyte
potentials at time $t=t^{n+1}$ are given by
$\Phi_k^h(x,t^{n+1}) = \sum_j \Phi_{k,j}(t^{n+1}) \varphi_{j,\Phi_k}(x)$, 
$k = 1, 2$.
$M$ is the mass matrix
\begin{equation}
    M =
    \begin{pmatrix}
    \langle \varphi_{i,\Phi_1} \vert aC \varphi_{j,\Phi_1} \rangle_\Omega & - \langle \varphi_{i,\Phi_1} \vert aC \varphi_{j,\Phi_2} \rangle_\Omega \\
    - \langle \varphi_{i,\Phi_2} \vert aC \varphi_{j,\Phi_1} \rangle_\Omega & \langle \varphi_{i,\Phi_2} \vert aC \varphi_{j,\Phi_2} \rangle_\Omega
    \end{pmatrix}
\end{equation}
The stiffness matrix $K$ is defined by
\begin{equation}
    K =
    \begin{pmatrix}
    \langle \nabla \varphi_{i,\Phi_1} \vert \sigma \nabla \varphi_{j,\Phi_1} \rangle_\Omega & 0 \\
    0 & \langle \nabla \varphi_{i,\Phi_2} \vert \kappa \nabla \varphi_{j,\Phi_2} \rangle_\Omega
    \end{pmatrix}
\end{equation}
The stiffness matrix may be modified to take into consideration the faradaic
processes (linearized version of the Butler-Volmer kinetics).
\newcommand{\pluseq}{\mathrel{+}=}
\begin{equation}
    K \pluseq
    \begin{pmatrix}
    \langle \varphi_{i,\Phi_1} \vert a \frac{a i_o  F (\alpha_a + \alpha_c)}{RT} \varphi_{j,\Phi_1} \rangle_\Omega & - \langle \varphi_{i,\Phi_1} \vert \frac{a i_o (\alpha_a + \alpha_c) F}{RT} \varphi_{j,\Phi_2} \rangle_\Omega \\
    - \langle \varphi_{i,\Phi_2} \vert \frac{a i_o  F (\alpha_a + \alpha_c)}{RT} \varphi_{j,\Phi_1} \rangle_\Omega & \langle \varphi_{i,\Phi_2} \vert \frac{a i_o (\alpha_a + \alpha_c) F}{RT} \varphi_{j,\Phi_2} \rangle_\Omega 
    \end{pmatrix}
\end{equation}
The right-hand side $f$ accounts for the prescribed solid current density on
the cathode current collector tab (current divided by tab surface area)
\begin{equation}
    f =
    \begin{pmatrix}
    \langle \varphi_{i,\Phi_1} \vert n \cdot \sigma \nabla \Phi_1 \rangle_{\partial\Omega} \\
    0
    \end{pmatrix}
\end{equation}

